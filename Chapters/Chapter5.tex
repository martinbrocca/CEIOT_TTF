% Chapter Template

\chapter{Conclusiones} % Main chapter title

\label{Chapter5} % Change X to a consecutive number; for referencing this chapter elsewhere, use \ref{ChapterX}


%----------------------------------------------------------------------------------------

%----------------------------------------------------------------------------------------
%	SECTION 1
%----------------------------------------------------------------------------------------


En este capítulo se muestran las conclusiones sobre el trabajo realizado. A su vez
se presentan algunas modificaciones o mejoras como posible trabajo futuro.
\section{Resultados obtenidos }


En este trabajo se completó el diseño, desarrollo, evaluación e implementación de un
prototipo de sistema escalable que automatiza el control y monitoreo de un invernadero inteligente de tipo hogareño, de acuerdo con el cronograma y los costos establecidos, al tiempo que se logró satisfacer todos los requerimientos del cliente.

Para su realización fueron de gran utilidad los conocimientos adquiridos a lo largo de la especialización, en particular resaltan:
\begin{itemize}
\item Gestión de proyectos: organizó y facilitó la ejecución de las tareas.
\item Protocolos de IoT: proveyó las bases para el entendimiento de la comunicación entre la capa física y la de aplicación.
\item Ciberseguridad en IoT: brindó herramientas y técnicas para la protección del sistema.
\item Desarrollo de aplicaciones para Internet de las cosas: dio los fundamentos para la programación del firmware y de los protocolos de comunicaciones. 
\end{itemize}



Durante la implementación del prototipo surgieron imprevistos que, si bien fueron resueltos, merecen ser mencionados:
\begin{itemize}
\item El desarrollo en escala introdujo complicaciones en los sistemas hidráulicos, por lo que se debieron emplear mangueras y conectores de alta presión para eliminar los riesgos de roturas.

\item El modelo de calibración para sensores capacitivos de humedad del suelo empleado  se basa en una relación lineal entre la capacitancia medida por el sensor y la humedad del suelo \citep{soilcalibration}. Sin embargo, esta ecuación no toma en cuenta otros factores que pueden afectar la medición, tales como la temperatura, la salinidad y la compactación del suelo. Por lo tanto, la precisión de la calibración puede verse comprometida si estos factores varían en el entorno en el que se encuentra el sensor.

Además, este modelo se basa en una configuración fija del dispositivo y no es fácilmente adaptable. Esto significa que si se desea introducir alguna modificación, resultará necesaria una nueva calibración, lo que puede ser costoso y requerir tiempo adicional.
\end{itemize}


\section{Conclusiones}
Las soluciones disponibles en el mercado suelen contar con una amplia gama de características integradas que pueden hacerlas más atractivas para quienes buscan una solución rápida y lista para usar. Sin embargo, el usuario queda supeditado a lo que ofrezca el producto y a las posibles ampliaciones determinadas por el fabricante.

Por el contrario, un invernadero inteligente como el realizado aporta flexibilidad, posibilidades de expansión y adaptabilidad a nuevos dispositivos que se desarrollen o adopten a futuro. Bastará con elegir y configurar los módulos sensores y dispositivos a utilizar en el sistema, lo que puede requerir un mayor esfuerzo y conocimientos técnicos.








%----------------------------------------------------------------------------------------
%	SECTION 2
%----------------------------------------------------------------------------------------
\section{Trabajo futuro}

Para dar continuidad al esfuerzo realizado hasta el momento y poder obtener un
producto atractivo surgen los siguientes puntos:

\begin{itemize}
\item Evaluar el rediseño o compra de los módulos físicos a fin de unificar los componentes electrónicos internos en una placa de circuito impreso más pequeña, considerando estándares de fabricación de placas electrónicas para uso comercial.

\item Configurar una red de tipo \textit{ad hoc} sobre la placa inalámbrica de la Raspberry PI y emplear la red Ethernet para el acceso a Internet, de forma de separar las redes personales de la del invernadero.

\item Evaluar diferentes opciones que permitan garantizar autenticación basada en certificados entre los dispositivos y la aplicación, como así también entre los usuarios y la aplicación para reducir exposiciones.

\item Continuar el análisis sobre la medición de humedad del suelo mediante sensores capacitivos, de forma de poder identificar mejoras o simplificaciones en la calibración del sistema conforme se varíe el tipo de suelo y de plantas empleadas. 

\item Expandir el sistema mediante la incorporación de nuevos actuadores, tales como tomacorrientes, que sean controlados por la aplicación y mediante los cuales pueda incorporar nuevos subsitemas entre los que se encuentran humidificadores y luces. 

\end{itemize}